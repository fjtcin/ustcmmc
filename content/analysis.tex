\section{问题分析}

本题中两个问题的主要目标都是使 \(\mathrm{BER}\) 的值尽可能小,因此我们先做整体上的分析。

首先我们假设已经设计好 \(V_1,\dots,V_m\),并设 \(V_i=\left\{v_{i1},\dots,v_{ia_i}\right\}\),
这里 \(a_i=\#V_i\) 是 \(V_i\) 的元素个数,满足 \(\sum_{i=1}^m a_i=2^n\)。
我们的目的是选择合适的 \(x_i\in V_i\),这也即要求发送 \(x_i\),接收到的信号仍在 \(V_i\) 内的概率最大。
为此,我们可以计算发送出 \(v_{ij}\) 而接收到 \(v_{il}\) 的概率 \(p^i_{jl}\),这里上标 \(i\) 表示考虑的集合是 \(V_i\)。
显然,\(p^i_{jl}=p^i_{lj}\)。
这样,发送 \(v_{ij}\) 后接收到的信号仍在 \(V_i\) 的概率为 \(P_{ij}=\sum_{l=1}^{a_i}p^i_{jl}\)。
取 \(x_i\) 为使得 \(P_{ij}\) 最大的 \(v_{ij}\) 即可。
于是 \(\mathrm{Prob}\left(x_j\ne x_i\right)=e_i=1-\max_{1\le j\le a_i}P_{ij}\),则 \(\mathrm{BER}=\max_{1\le i\le m}e_i\)。

从以上分析可以看出,一旦设计好 \(V_1,\dots,V_m\),则可以设计出合适的 \(x_1,\dots,x_m\)。
现在需要考虑如何设计 \(V_1,\dots,V_m\)。