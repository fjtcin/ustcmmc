\section{问题重述}

通信系统在当今社会中扮演了十分重要的角色。
其中,信息的传递是传输信息串实现的,而每一个信息串由若干个比特(0 或 1)组成。
显然,由于实际环境中的噪音,信息传递不可能完全准确。
在本题中,我们只考虑一种较为简单的噪音:比特是通过二元对称信道传输的。
在二元对称信道中,发送一个比特,接收到的比特有概率 \(p\) 与原来不同。
假定 \(p\in(0,1)\) 是一个常数,且每一个比特的发送和接收是独立的。

设 \(V=\left\{0,1\right\}^n\) 是含 \(n\) 个比特的信息全体。
给定 \(k<n\),划分 \(V\) 为 \(m=2^k\) 个集合 \(V_1,V_2,\dots,V_m\),
即 \(V_i\) 两两不交,且 \(\bigcup_{i=1}^m V_i=V\)。
对每个 \(V_i\),选取一个 \(x_i\) 作为其代表。
以后,我们仅发送这些选定的代表。若发送$x_i$,接收到的信息为$y$,则解码为$y$所在集合$V_j$的代表$x_j$。
记 \(e_i\) 为“错误解码”的概率,即 \(e_i=\mathrm{Prob}\left(x_j\ne x_i\right)\)。

定义 \(\mathrm{BER}=\max_{1\le i\le m}e_i\)。我们需解决以下两个问题:
\begin{enumerate}[nosep]
  \item 对给定的 \(r=k/n\),设计 \(V_1,\dots,V_m\) 及代表 \(x_1,\dots,x_m\),使得 \(\mathrm{BER}\) 尽可能小。
  \item 设计一套算法,对输入的 \(n\) 和 \(k\) 能够给出相应的 \(V_1,\dots,V_m\) 和 \(x_1,\dots,x_m\),
  使得 \(\mathrm{BER}\) 尽可能小。以 \(k=24\),\(n=32\),\(p=0.1\) 为例进行分析。
\end{enumerate}