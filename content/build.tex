\section{模型的建立与求解}

\subsection{问题一的求解}

我们将给出BER最小值的一个估计。

\begin{lemma}
代数式$\binom{n}{q}p^q(1-p)^{n-q}$的最大值在$q=\lceil (n+1)p\rceil-1$时取得。
\end{lemma}
\begin{proof}
记$f(q)=\binom{n}{q}p^q(1-p)^{n-q}$,则
$$
\frac{f(q+1)}{f(q)}=\frac{p(n-q)}{(1-p)(q+1)}
$$
由此可知$q\le (n+1)p-1$时$f(q)$单调增加,否则$f(q)$单调减少。引理得证。
\end{proof}

记$q_0=\lceil (n+1)p\rceil-1$,$p_0=\binom{n}{q_0}p^{q_0}(1-p)^{n-q_0}$。因为任何$p^i_{jl}$不超过$p_0$,于是我们可以给出BER的最小值的一个下界。
\begin{align*}
    \mathrm{BER}\text{的最小值}&=\max_{1\le i\le m}e_i\\
    &\ge\frac{\sum_{i=1}^m e_i}{m}\\
    &=\frac{m-\sum_{i=1}^m\sum_{l=1}^{a_i}p^i_{jl}}{m}\\
    &\ge\frac{m-p_0\sum_{i=1}^m a_i}{m}\\
    &=\frac{m-p_0\cdot2^n}{m}\\
    &=1-p_0\cdot 2^{n-k}
\end{align*}

下面来估计BER的最小值的上界。我们取$V$的一个划分,使得诸$e_i$相等。取$t$是使得$2\sum_{l=1}^t\binom{n}{l}\le 2^{n-k}$的最大整数,则显然$t\le n/2$。通过估计各$f(s)$的个数可得
$$
e_i\le 1-\sum_{s=0}^t\binom{n}{s}\left(p^s(1-p)^{n-s}+p^{n-s}(1-p)^s\right)
$$

故
$$
\mathrm{BER}\text{的最小值}\le e_i\le 1-\sum_{s=0}^t\binom{n}{s}\left(p^s(1-p)^{n-s}+p^{n-s}(1-p)^s\right)
$$
