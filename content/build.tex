\section{模型的建立与求解}

\subsection{问题一的求解}

我们将给出 \(\mathrm{BER}\) 最小值的一个估计。

\noindent\fcolorbox{black}{white}{引理} 代数式 \(\binom{n}{q}p^q(1-p)^{n-q}\) 的最大值在 \(q=\left\lceil\left(n+1\right)p\right\rceil-1\) 时取得。

\noindent\fcolorbox{white}{white}{证明} 记 \(f\left(q\right)=\binom{n}{q}p^q\left(1-p\right)^{n-q}\), 则
\[\frac{f\left(q+1\right)}{f\left(q\right)}=\frac{p\left(n-q\right)}{\left(1-p\right)\left(q+1\right)}\]
由此可知 \(q\le\left(n+1\right)p-1\) 时 \(f\left(q\right)\) 单调增加,否则 \(f\left(q\right)\) 单调减少。\qed

记 \(q_0=\left\lceil\left(n+1\right)p\right\rceil-1\), \(p_0=\binom{n}{q_0}p^{q_0}(1-p)^{n-q_0}\)。
因为任何 \(p^i_{jl}\) 不超过 \(p_0\), 于是我们可以给出 \(\mathrm{BER}\) 的最小值的一个下界。
\begin{align*}
  \mathrm{BER}_\mathrm{min}&=\max_{1\le i\le m}e_i\\
  &\ge\frac{\sum_{i=1}^m e_i}{m}\\
  &=\frac{m-\sum_{i=1}^m\sum_{l=1}^{a_i}p^i_{jl}}{m}\\
  &\ge\frac{m-p_0\sum_{i=1}^m a_i}{m}\\
  &=\frac{m-p_0\cdot2^n}{m}\\
  &=1-p_0\cdot 2^{n-k}
\end{align*}

下面来估计 \(\mathrm{BER}\) 的最小值的上界。
我们取 \(V\) 的一个划分,使得诸 \(e_i\) 相等。
取 \(t\) 是使得 \(2\sum_{l=1}^t\binom{n}{l}\le 2^{n-k}\) 的最大整数,则显然 \(t\le n/2\)。
通过估计各 \(f\left(s\right)\) 的个数可得
\[e_i\le 1-\sum_{s=0}^t\binom{n}{s}\left(p^s\left(1-p\right)^{n-s}+p^{n-s}\left(1-p\right)^s\right)\]

故
\[\mathrm{BER}_\mathrm{min}\le e_i\le 1-\sum_{s=0}^t\binom{n}{s}\left(p^s\left(1-p\right)^{n-s}+p^{n-s}\left(1-p\right)^s\right)\]

\subsection{问题二的求解}

由第一问
\[1-p_0\cdot 2^{n-k}\le\mathrm{BER}_\mathrm{min}\le 1-\sum_{s=0}^t\binom{n}{s}\left(p^s\left(1-p\right)^{n-s}+p^{n-s}\left(1-p\right)^s\right)\]
通常 \(\mathrm{BER}_\mathrm{min}\) 的上下界相差不大,随机方法有很好的效果,算法如下:
\begin{enumerate}
  \item 定义数池的初始值 \(\mathrm{Pool}=V=\left\{0,1\right\}^n\)
  \item 在 \(\left[l,r\right]\) 区间内二分 \(\mathrm{BER}_\mathrm{min}\), 二分值记为 \(m\)
  \begin{enumerate}
    \item 从数池 \(\mathrm{Pool}\) 中随机取出 \(2^{n-k}\) 个数形成一组 \(V_i\)
    \item 计算这组 \(V_i\) 的 \(e_i\)
    \item 如果 \(e_i>m\), 则清空所有分好的组,从头重新随机分配。
    \item 若随机分配了若干次,仍有 \(e_i>m\), 则在 \(\left[m,r\right]\) 继续二分;否则在 \(\left[l,m\right]\) 继续二分
    \item 当 \(l,r\) 相近时 \(\left(r-l<\mathrm{eps},\text{ 其中 }\mathrm{eps}\text{ 是一个小量}\right)\), 则可退出二分
  \end{enumerate}
  \item \(\mathrm{BER}_\mathrm{min}=r\), \(V\) 的分组即为相应的随机分配结果。
\end{enumerate}

\(\mathrm{Pool}\) 是一个数据结构,为优化时间复杂度,其要满足:
\begin{enumerate}
  \item \(O\left(1\right)\) 插入、删除
  \item \(O\left(1\right)\) 返回其中的一个随机数值
\end{enumerate}

实现方法是\cite{main},考虑一个由哈希表 H 和数组 A 组成的数据结构。哈希表的键是数据结构中的元素,值是它们在数组中的位置。
\begin{enumerate}
  \item insert(value): 将值附加到数组中,并让 i 成为它在 A 中的索引。设置 H[value]=i。
  \item remove(value):我们将用 A 中的最后一个元素替换 A 中包含 value 的单元格。设 d 是数组 A 中的最后一个元素。设 i 为 H[value],即要删除的值在数组中的索引。设置 A[i]=d, H[d]=i,将数组大小减一,并从 H 中删除 value(令 H[value]=0)。
  \item getRandomElement():让 r=random(A.size())。返回 A[r]。
\end{enumerate}

由上,时空复杂度均为 \(O\left(2^n\right)\), 时间复杂度中具体的常数大小需要由随机的次数决定。