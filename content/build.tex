\section{模型的建立与求解}

\subsection{问题一的求解}

我们将给出 \(\mathrm{BER}\) 最小值的一个估计。

\noindent\fcolorbox{black}{white}{引理} 代数式 \(\binom{n}{q}p^q(1-p)^{n-q}\) 的最大值在 \(q=\left\lceil\left(n+1\right)p\right\rceil-1\) 时取得。

\noindent\fcolorbox{white}{white}{证明} 记 \(f\left(q\right)=\binom{n}{q}p^q\left(1-p\right)^{n-q}\), 则
\[\frac{f\left(q+1\right)}{f\left(q\right)}=\frac{p\left(n-q\right)}{\left(1-p\right)\left(q+1\right)}\]
由此可知 \(q\le\left(n+1\right)p-1\) 时 \(f\left(q\right)\) 单调增加,否则 \(f\left(q\right)\) 单调减少。\qed

记 \(q_0=\left\lceil\left(n+1\right)p\right\rceil-1\), \(p_0=\binom{n}{q_0}p^{q_0}(1-p)^{n-q_0}\)。
因为任何 \(p^i_{jl}\) 不超过 \(p_0\), 于是我们可以给出 \(\mathrm{BER}\) 的最小值的一个下界。
\begin{align*}
  \mathrm{BER}_\mathrm{min}&=\max_{1\le i\le m}e_i\\
  &\ge\frac{\sum_{i=1}^m e_i}{m}\\
  &=\frac{m-\sum_{i=1}^m\sum_{l=1}^{a_i}p^i_{jl}}{m}\\
  &\ge\frac{m-p_0\sum_{i=1}^m a_i}{m}\\
  &=\frac{m-p_0\cdot2^n}{m}\\
  &=1-p_0\cdot 2^{n-k}
\end{align*}

下面来估计 \(\mathrm{BER}\) 的最小值的上界。
我们取 \(V\) 的一个划分,使得诸 \(e_i\) 相等。
取 \(t\) 是使得 \(2\sum_{l=1}^t\binom{n}{l}\le 2^{n-k}\) 的最大整数,则显然 \(t\le n/2\)。
通过估计各 \(f\left(s\right)\) 的个数可得
\[e_i\le 1-\sum_{s=0}^t\binom{n}{s}\left(p^s\left(1-p\right)^{n-s}+p^{n-s}\left(1-p\right)^s\right)\]

故
\[\mathrm{BER}_\mathrm{min}\le e_i\le 1-\sum_{s=0}^t\binom{n}{s}\left(p^s\left(1-p\right)^{n-s}+p^{n-s}\left(1-p\right)^s\right)\]